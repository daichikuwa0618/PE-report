\documentclass[a4paper,12pt, uplatex, fleqn]{jsarticle}

% 余白の設定
\setlength{\textwidth}{\fullwidth}
\setlength{\textheight}{40\baselineskip}
\addtolength{\textheight}{\topskip}
\setlength{\voffset}{-0.2in}
\setlength{\topmargin}{0pt}
\setlength{\headheight}{0pt}
\setlength{\headsep}{0pt}

% パッケージ
\usepackage[dvipdfmx]{graphicx}     % 画像
\usepackage{amsmath, amssymb}       % ギリシャ文字
\usepackage{bm}                     % 数式
\usepackage{comment}                % コメント
\usepackage{siunitx}                % SI単位
\usepackage{framed}                 % 枠組み
\usepackage{color}                  % 色の使用
\usepackage{type1cm}                % エラー対策
\usepackage{cases}                  % 行列

\begin{document}
\section{行動による心拍数の変化}
    記録をとった1週間の行動を以下のように分類した。
    \begin{enumerate}
        \item 朝食
        \item 昼食
        \item 夕食
        \item 授業
        \item 受験勉強
        \item 休憩
        \item 卒業研究
        \item 運動(体育も含む)
        \item 休憩 (自由時間)
        \item シャワー (お風呂)
        \item 睡眠中
    \end{enumerate}

    以下に測定結果からまとめた表を示す。また、行動がない場合や、測定ができていない箇所は空欄とした。

    \begin{table}[htbp]
        \begin{center}
            \caption{行動による心拍数の変化}
            \begin{tabular}{|l||r|r|r|r|r||r|} \hline
                & 7月25日 & 7月26日 & 7月28日 & 7月29日 & 7月31日 & 平均\\ \hline \hline
                朝食 &  & 65 & 73 & 50 & 60 & 62.0 \\ \hline
                昼食 & 80 & 85 & 80 & 85 & 100 & 86.0 \\ \hline
                夕食 & 82 & 72 & 78 & 65 & 80 & 75.4 \\ \hline
                授業 & 75 & 85 & 80 &  & 100 & 85.0\\ \hline
                受験勉強 & 80 & 85 & 75 & 70 & 73 & 76.6 \\ \hline
                休憩 & 85 & 90 & 90 &  & 110 & 93.8 \\ \hline
                卒業研究 &  & 85 & 75 &  & 80 & 80.0 \\ \hline
                運動 &  &  &  &  & 145 & 145 \\ \hline
                自由時間 & 78 & 70 &  & 80 & & 76.0 \\ \hline
                シャワー & 82 & 75 & 90 & 88 & 70 & 81.0 \\ \hline
                睡眠中 & 50 & 51 & 48 & 50 & 48 & 49.4 \\ \hline
            \end{tabular}
        \end{center}
    \end{table}
    \newpage

\section{授業による心拍数の変化}
    授業ごとの心拍数は以下のようになった。
    \begin{table}[htbp]
        \begin{center}
        \begin{tabular}{ccccc}
            計算機:85 & ロボット:80 & シムテム:80 & マイコン:89 & 材料:77 \\
            工業数学:90 & 電子制御設計:80 & 環境化学:85 & 体育:145 & 中国語:102 \\
        \end{tabular}

        \end{center}
    \end{table}

    まず、体育は体を動かしており、運動強度の高いバスケットボールをしたことが心拍数が高くなった要因としてあげられる。次に、この中で計算機・マイコン・工業数学・体育は1限目であり、そのほかと比べると5〜10ほど心拍数が高い。これは被験者が遅刻気味であり、いつも焦って投稿しているためではないかと思う。\\
    最後に中国語が2番目に高い要因としては、この授業の体制にあるのではないかと思う。この授業はとにかく発生が多く、学生同士の発音練習などとにかく発言する機会が多い。このことが要因ではないかと思う。


\section{日常の運動を継続することによる影響}
    私は普段ランニングを行なっている。この期間は受験期間で実施していなかったが、競技としてだけでなく、趣味としても定期的に走っている。今回はデータがないが、ランニングについて述べようと思う。\\
    まず、結論としてランニングを継続することによって得られる効果は以下に述べることがあると思う。

    \begin{enumerate}
        \item 肺活量の増加
        \item 心拍機能の向上
        \item 毛細血管の増加
        \item 体幹の筋力増強
    \end{enumerate}

    それぞれについて以下に説明する
    \subsection{肺活量の増加}
        ランニングは有酸素運動であり、他のスポーツと比較して長時間行われるため、運動強度としては高くなくても常に普段以上の酸素が必要となる。そのため呼吸回数も増え、1回の呼吸で吸気・排気する空気の量も増える。そのため肺胞が大きくなり、結果として呼吸機能が強化する。

    \subsection{心拍機能の向上}
        肺活量の部分でも述べたようにランニングは長時間行われる。そのため常に心拍数は自分の場合140を超える。このような運動を続けると心臓の筋肉が発達し、1回の心拍で送ることができる血液の量が増える。そのため結果として心拍数は減少する。

    \subsection{毛細血管の増加}
        毛細血管についてはどのスポーツでも言えることではあるが、ランニングは顕著に効果があると思う。陸上競技のトレーニングにLSD(Long Slow Distance)というものがあるが、これは会話ができるほどの遅いペースで目安として120分以上走るというものがある。このトレーニングの目的はまさに毛細血管の増加である。なぜランニングで毛細血管が増えるかというと末端まで血管を張り巡らすことでより効率よく二酸化炭素を血管中に、血管から酸素を細胞にとガス交換を行うためである。

    \subsection{体幹の筋力増強}
        ランニングは長時間同じ姿勢を保つ運動である。上半身が動いてぶれると足にかかる負担がバラバラになり、効率よく走ることができずすぐに疲れてしまう。ランニングの時の上半身は地面に対して垂直かほんの少し前傾になりその姿勢を保ち続ける。そのためこの運動を継続すれば自然と上半身を支える体幹部分が強くなる。

\section{エネルギー消費量の考察}
    日毎のエネルギー消費量をまとめて示す。
    \begin{table}[htbp]
        \begin{center}
        \caption{日毎のエネルギー消費量}
            \begin{tabular}{l|ccccc}
                日付 & 7/25 & 7/26 & 7/28 & 7/29 & 7/31 \\ \hline
                消費量[kcal] & 3176 & 3130 & 3339 & 3352 & 4283 \\
            \end{tabular}
        \end{center}
    \end{table}

\section{深睡眠時間の考察}
日毎の深睡眠時間をまとめて示す。
    \begin{table}[htbp]
        \begin{center}
        \caption{日毎の深睡眠時間}
            \begin{tabular}{l|ccccc}
                日付 & 7/25 & 7/26 & 7/28 & 7/29 & 7/31 \\ \hline
                時間[hour] & 5.02 & 3.93 & 4.03 & 2.77 & 4.04 \\
            \end{tabular}
        \end{center}
    \end{table}

\section{結果から最適な運動の提案}
今回の結果から、自分の日常生活に運動を組み込むとしたら、以下の条件を満たしたものが理想ではないかと考える
\begin{enumerate}
    \item 楽しんで行える
    \item 運動強度は強すぎず長時間でも行える
    \item 場所・時間(曜日)にとらわれない
    \item 思い立ったらすぐに行うことができる
    \item 身体的に良い影響がある
\end{enumerate}
これらの条件を踏まえると結論として理想のスポーツはウォーキング・ランニング(競技としてではない)・ダンスではないかと思う。
しかし、スポーツの根底として「楽しむ」ことが大前提なので、スポーツは本人がやりたい種目を選択するべきだと思う。\\
大切なのは何をするかではなく、そのスポーツの運動強度などの要素をしっかりと見極め、運動頻度・1回の運動時間・休憩時間・運動をする時間帯などを自分で考えることだと今回の実験で感じた。


\end{document}
