\documentclass[a4paper,12pt, uplatex, fleqn]{jsarticle}

% 余白の設定
\setlength{\textwidth}{\fullwidth}
\setlength{\textheight}{40\baselineskip}
\addtolength{\textheight}{\topskip}
\setlength{\voffset}{-0.2in}
\setlength{\topmargin}{0pt}
\setlength{\headheight}{0pt}
\setlength{\headsep}{0pt}

% パッケージ
\usepackage[dvipdfmx]{graphicx}     % 画像
\usepackage{amsmath, amssymb}       % ギリシャ文字
\usepackage{bm}                     % 数式
\usepackage{comment}                % コメント
\usepackage{siunitx}                % SI単位
\usepackage{framed}                 % 枠組み
\usepackage{color}                  % 色の使用
\usepackage{type1cm}                % エラー対策
\usepackage{cases}                  % 行列

% 表紙の中身
\makeatother
\title{}
\date{}
\id{13332}
\department{}
\author{林大地}

\begin{document}
\section{行動による心拍数の変化}
記録をとった1週間の行動を以下のように分類した。
\begin{enumerate}
    \item 朝食
    \item 昼食
    \item 夕食
    \item 授業
    \item 受験勉強
    \item 休憩
    \item 卒業研究
    \item 運動(体育も含む)
    \item 休憩 (自由時間)
    \item シャワー (お風呂)
    \item 睡眠中
\end{enumerate}

以下に測定結果からまとめた表を示す。また、行動がない場合や、測定ができていない箇所は空欄とした。

\begin{conter}
    \begin{tabular}{|l|rrrrr|} \hline
        & 7月25日 & 7月26日 & 7月28日 & 7月29日 & 7月31日 \\ \hline
        朝食 &  & 65 & 73 & 50 & 60 \\ \hline
        昼食 & 80 & 85 & 80 & 85 & 100 \\ \hline
        夕食 & 82 & 72 & 78 & 65 & 80 \\ \hline
        授業 & 75 & 85 & 80 &  & 100 \\ \hline
        受験勉強 & 80 & 85 & 75 & 70 & 73 \\ \hline
        休憩 & 85 & 90 & 90 &  & 110 \\ \hline
        卒業研究 &  & 85 & 75 &  & 80 \\ \hline
        運動 &  &  &  &  & 145 \\ \hline
        自由時間 & 78 & 70 &  & 80 &  \\ \hline
        シャワー & 82 & 75 & 90 & 88 & 70 \\ \hline
        睡眠中 & 50 & 51 & 48 & 50 & 48 \\ \hline
    \end{tabular}
\end{conter}

\section{授業による心拍数の変化}


\section{日常の運動を継続することによる影響}

\section{エネルギー消費量の考察}

\section{深睡眠時間の考察}

\section{結果から最適な運動の提案}


\end{document}
